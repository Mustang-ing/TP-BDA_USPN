\documentclass{rapport}


\usepackage{lipsum}
\usepackage{tikz} 
\usepackage{listings}
\usepackage{amsfonts} 
\usepackage{listings}
\usepackage[T1]{fontenc}
\usepackage{color}
\usepackage{setspace}
\usepackage{listings}
\usepackage{xcolor}
\usepackage{tcolorbox}
\usepackage[utf8]{inputenc}
\usepackage[T1]{fontenc}
\usepackage[french]{babel}
\usepackage[footnote]{acronym}
\usepackage{fancyhdr}
\usepackage{helvet}
\usepackage{svg}
\usepackage{float}      % Add this to your preamble
\usepackage{placeins}   % Ensures floats do not cross section boundaries
\usepackage{caption}



\pagestyle{fancy}
\fancyhead{}
\lhead{\bfseries\leftmark}
\cfoot{\bfseries\thepage}
\setlength{\headheight}{14.7pt}
\renewcommand{\headrulewidth}{1pt}
\renewcommand{\footrulewidth}{1pt}


\definecolor{dkgreen}{rgb}{0,0.6,0}
\definecolor{gray}{rgb}{0.5,0.5,0.5}
\definecolor{mauve}{rgb}{0.58,0,0.82}

\renewcommand{\thesection}{\arabic{section}} % Sections: 1, 2, 3...
\renewcommand{\thesubsection}{\thesection.\arabic{subsection}} % Subsections: 1.1, 1.2...
\renewcommand{\thesubsubsection}{\thesubsection.\arabic{subsubsection}} % Subsubsections: 1.1.1, 1.1.2...

\lstset{
    basicstyle=\ttfamily\small,
    keywordstyle=\color{blue},
    commentstyle=\color{gray},
    stringstyle=\color{red},
    breaklines=true,
    frame=single,
    captionpos=b,
}

\title{Rapport TP2} %Titre du fichier

\doublespacing 
\begin{document}
\lhead{Base de données avancée \textbf{- \titre}}

%----------- Informations du Rapport --------
\logo{logos/logo2.png}
\unif{}
\titre{Rapport TP3 - Dévelloppement d'un micro-services }
\sujet{ Base de données avancée}
%----------- Inserer l'image-logo içi---------
\enseignant{ Samir \textsc{Youssef} } %Nom de l'enseignant


\eleve{ Aïssa \textsc{Pansan} \\ 11933936}

\newpage

%----------- Initialisation -------------------

\renewcommand*\contentsname{Sommaire}       
\makenomenclature %Afficher les marges
\fairepagedegarde %Créer la page de garde


%------------ Corps du rapport ----------------



%------ Proto intégration image -------
\iffalse
\begin{figure}[H]
	\centering
    \includegraphics[width=15cm]{images/Config_1_3.png}
    \caption{Configuration}
\end{figure}
\FloatBarrier

%Minipage 

\begin{minipage}{0.5\textwidth}
\includegraphics[width=8cm]{images/P1/Q4_a1.png}
\captionof{figure}{Tentative de lister le contenu}
\label{fig:figure}
\end{minipage}
\hspace{1cm}
\begin{minipage}{0.5\textwidth}
\includegraphics[width=8cm]{images/P1/Q4_b.png}
\captionof{figure}{Listing avec \textbf{info2}}
\label{fig:figure}
\vspace{1cm}
\end{minipage}


\fi

\section{Initialisation}

Pour réaliser une micro-service en SpringBoot, on peut utiliser plusieurs IDE comme \textbf{IntelliJIdea}. Cela dit, cela n'a pas marché pour (malgré pluseurs essai, je vous le promets). \textbf{Dans la suite de ce rapport, l'ensemble des opérations seront effectué dans l'IDE Eclipse}.\newline

Les premières êtapes sont les même que celle de la vidéo, on peut les faire sur \textbf{IntelliJ} à la différence du choix des versions.

\begin{itemize}
	\item On choisit la version 21 de Java.
	\item On utilise le packagink Jar, pour déployer sans un serveur d'application. C'est comme un zip. 
	\item On doit utiliser la dépendance \textbf{Spring Web}, on peut enfin valider.
\end{itemize}
Remarque :  Un projet Spring est un projet Maven\newline

Ensuite, en ligne de commande. On va initialiser la projet Maven via la commande \textbf{mvn install} 

C'est à partir de cette êtape, que l'on utilisera Eclipse. 


\section{Lancement}

SpringBot intègre directement Tomcat. On n'a pas besoin de paramétré l'IDE pour lancer Tomcat, ni même l'installer. On va lancer notre application \textbf{src/main/java/com/example/}. On accède à l'application via localhost sur le port 8080

\begin{figure}[H]
	\centering
    \includegraphics[width=15cm]{images/P1.png}
    \caption{Illustration du lancement de SpringBoot}
\end{figure}

Les micros-services doivent être écrit au niveau de \textbf{com.example.[Nom_projet]}. 

\begin{figure}[H]
	\centering
    \includegraphics[width=15cm]{images/P2_A.png}
    \caption{Localisation dans Eclipse}
\end{figure}




\section{Pourquoi cela ne marche sur IntelliJ ?}

J'ai décidé de basculer sur 

\end{document}






























