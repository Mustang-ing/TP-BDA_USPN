\documentclass{rapport}


\usepackage{lipsum}
\usepackage{tikz} 
\usepackage{listings}
\usepackage{amsfonts} 
\usepackage{listings}
\usepackage[T1]{fontenc}
\usepackage{color}
\usepackage{setspace}
\usepackage{listings}
\usepackage{xcolor}
\usepackage{tcolorbox}
\usepackage[utf8]{inputenc}
\usepackage[T1]{fontenc}
\usepackage[french]{babel}
\usepackage[footnote]{acronym}
\usepackage{fancyhdr}
\usepackage{helvet}
\usepackage{svg}
\usepackage{float}      % Add this to your preamble
\usepackage{placeins}   % Ensures floats do not cross section boundaries
\usepackage{caption}
\usepackage{hyperref}



\pagestyle{fancy}
\fancyhead{}
\lhead{\bfseries\leftmark}
\cfoot{\bfseries\thepage}
\setlength{\headheight}{14.7pt}
\renewcommand{\headrulewidth}{1pt}
\renewcommand{\footrulewidth}{1pt}


\definecolor{dkgreen}{rgb}{0,0.6,0}
\definecolor{gray}{rgb}{0.5,0.5,0.5}
\definecolor{mauve}{rgb}{0.58,0,0.82}

\renewcommand{\thesection}{\arabic{section}} % Sections: 1, 2, 3...
\renewcommand{\thesubsection}{\thesection.\arabic{subsection}} % Subsections: 1.1, 1.2...
\renewcommand{\thesubsubsection}{\thesubsection.\arabic{subsubsection}} % Subsubsections: 1.1.1, 1.1.2...

\lstset{
    basicstyle=\ttfamily\small,
    keywordstyle=\color{blue},
    commentstyle=\color{gray},
    stringstyle=\color{red},
    breaklines=true,
    frame=single,
    captionpos=b,
}

\title{Rapport TP2} %Titre du fichier

\doublespacing 
\begin{document}
\lhead{Base de données avancée \textbf{- \titre}}

%----------- Informations du Rapport --------
\logo{logos/logo2.png}
\unif{}
\titre{Rapport TP3 - Dévelloppement d'un micro-services }
\sujet{ Base de données avancée}
%----------- Inserer l'image-logo içi---------
\enseignant{ Samir \textsc{Youssef} } %Nom de l'enseignant


\eleve{ Aïssa \textsc{Pansan} \\ 11933936}

\newpage

%----------- Initialisation -------------------

\renewcommand*\contentsname{Sommaire}       
\makenomenclature %Afficher les marges
\fairepagedegarde %Créer la page de garde


%------------ Corps du rapport ----------------



%------ Proto intégration image -------
\iffalse
\begin{figure}[H]
	\centering
    \includegraphics[width=15cm]{images/Config_1_3.png}
    \caption{Configuration}
\end{figure}
\FloatBarrier

%Minipage 

\begin{minipage}{0.5\textwidth}
\includegraphics[width=8cm]{images/P1/Q4_a1.png}
\captionof{figure}{Tentative de lister le contenu}
\label{fig:figure}
\end{minipage}
\hspace{1cm}
\begin{minipage}{0.5\textwidth}
\includegraphics[width=8cm]{images/P1/Q4_b.png}
\captionof{figure}{Listing avec \textbf{info2}}
\label{fig:figure}
\vspace{1cm}
\end{minipage}


\fi

\section{Initialisation}

Pour réaliser une micro-service en SpringBoot, on peut utiliser plusieurs IDE comme \textbf{IntelliJIdea}. Cela dit, cela n'a pas marché pour (malgré pluseurs essai, je vous le promets). \textbf{Dans la suite de ce rapport, l'ensemble des opérations seront effectué dans l'IDE Eclipse}.\newline

Les premières êtapes sont les même que celle de la vidéo, on peut les faire sur \textbf{IntelliJ} à la différence du choix des versions.

\begin{itemize}
	\item On choisit la version 21 de Java.
	\item On utilise le packagink Jar, pour déployer sans un serveur d'application. C'est comme un zip. 
	\item On doit utiliser la dépendance \textbf{Spring Web}, on peut enfin valider.
\end{itemize}
Remarque :  Un projet Spring est un projet Maven\newline

Ensuite, en ligne de commande. On va initialiser la projet Maven via la commande \textbf{mvn install} 

C'est à partir de cette êtape, que l'on utilisera Eclipse. 


\section{Écriture d'une première API REST}

SpringBot intègre directement Tomcat. On n'a pas besoin de paramétré l'IDE pour lancer Tomcat, ni même l'installer. On va lancer notre application \textbf{src/main/java/com/example/}. On accède à l'application via localhost sur le port 8080

\begin{figure}[H]
	\centering
    \includegraphics[width=15cm]{images/P1.png}
    \caption{Illustration du lancement de SpringBoot}
\end{figure}

Les micros-services doivent être écrit au niveau de \textbf{com.example.[Nom\_projet]}. 

\begin{figure}[H]
	\centering
    \includegraphics[width=15cm]{images/P2_A.png}
    \caption{Localisation dans Eclipse}
\end{figure}

On définit une nouvelle app via deux annotations primordiales.
\begin{itemize}
	\item \textbf{@RestController} : Pour définir une API REST. Elle se met au niveau de la nouvelle classe
	\item \textbf{@GetMapping(value = "/bn")} : Comme une Servlet. On définit la HTTP Response en précisant l'URL d'accès. Chaque méthode peut représenter une page.
\end{itemize}

\textcolor{red}{\textbf{Attention} : Il faut arrêter la SpringBootApplication pour appliquer les changements. Relancer l'app, alors quelle est déjà lancé ne marchera pas. Il faudra relancer Eclipse pour arreter l'app une bonne fois pour toutes. Pour eviter tout ça arrete l'app via la console.}

\begin{figure}[H]
	\centering
    \includegraphics[width=15cm]{images/P2_C.png}
    \caption{Arret de l'application via la console}
\end{figure}

\begin{figure}[H]
	\centering
    \includegraphics[width=15cm]{images/P2_D.png}
    \caption{Résultats}
\end{figure}
\subsection{Changer le port d'écoute de l'application}

Pour changer le port d'écoute. On modifie le fichier dans \textbf{src/main/ressources/main.properties} et on écrit : \textbf{server.port=9999}


\subsection{Afficher les informations d'un objet}

On va générer une nouvelle classe Étudiant avec les attributs id,nom et moyenne. Jusqu'a là c'est du Java classique (Constructeur par défaut, Constructeur normal, Getters et Setters).
On peut afficher les attributs d'un objet en retournant directement l'objet d'un classe dans une méthode qu'on mappe. L'affichage se fera sous format JSON.

\begin{figure}[H]
	\centering
    \includegraphics[width=15cm]{images/P3.png}
    \caption{Résultats}
\end{figure}

\section{Tester les micro-services}

Plusieurs outils existent pour tester un micro-services (PostMan,DeepLogin,SoapUI). Dans notre cas, je vais utiliser SOAPUI, car il est déjà présent dans ma machine.\newline

Pour tester un micro-services. On va écrire un fonction avec l'entrypoint(\textbf{/somme} qui retourn la somme de deux nombres fourni en paramètre. Au niveau de l'URL, on doit passé les deux paramètre comme-ci : \verb| \textbf{ /somme?a=1&b=1}|. On peut tester manuellement via le navigateur, mais SOAPUI nous permer de le faire via une interface graphique.

\verb|foo_bar|

\subsection{Etape 1 : Choisir un projet REST}

L'URI est le début de l'URL, c'est à dire : http://localhost

\begin{figure}[H]
	\centering
    \includegraphics[width=15cm]{images/P4_A.png}
    \caption{On ne précise pas encore le port}
\end{figure}

\subsection{Etape 2 : Tester}

\begin{figure}[H]
	\centering
    \includegraphics[width=15cm]{images/P4_B.png}
    \caption{Interface SoapUI}
\end{figure}

Il y a trois champs important dans SoapUI :
\begin{itemize}
	\item \textbf{Entrypoint :} Le point d'entrée de l'app. C'est le début de l'URL
	\item \textbf{Ressources : } C'est la page spécifique auquel on veut accéder
	\item \textbf{Parameters : } Les paramètre à envoyer (HTTP Response)
\end{itemize}
On peut sauvegarder au fur et à mesure pour garder les paramètres de la requêtes.

\textcolor{red}{\textbf{Remarque} : Les requêtes sont en GET}

\section{Persistance des informations via une ArrayList}

Une étape phare d'une API REST est le fait de sauvegarder de manière pérenne les informations. Pour le moment, on va utiliser une collection d'étudiant statique qui sera stocké dans une ArrayList.\newline
\begin{lstlisting}
public static Collection<Etudiant> liste = new ArrayList();
\end{lstlisting}

On va ensuite ajouter au fur et à mesure des étudiants dans la liste via \textbf{liste.add( )}. Tout se fait en static.

Enfin, on fait une nouvelle méthode avec l'entrypoint \textbf{/list} qui retourne la collection. 


\begin{figure}[H]
	\centering
    \includegraphics[width=15cm]{images/P5.png}
    \caption{Affichage des éleves en JSON}
\end{figure}


\subsection{Affichage d'un étudiant via son identifiant}
On peut retourner un éléve en fonction de son identifiant. En faisant, une nouvelle méthode \textbf{getEtudiant}, on doit préciser l'id de l'étudiant via l'URL (Comme auparvant). Si l'étudiant existe, on peut voir ses informations sinon, il y a une erreur.



\begin{minipage}{0.5\textwidth}
\includegraphics[width=8cm]{images/P6_A.png}
\captionof{figure}{Etudiant 0}
\label{fig:figure}
\end{minipage}
\hspace{1cm}
\begin{minipage}{0.5\textwidth}
\includegraphics[width=8cm]{images/P6_B.png}
\captionof{figure}{Etudiant introuvable}
\label{fig:figure}
\vspace{1cm}
\end{minipage}

\subsection{Ajouter des étudiants}

Rappel des type de requête HTTP : 
\begin{itemize}
	\item GET : Renvoie une ressources
	\item POST : Générer une ressources
	\item PUT : Modifier une ressources 
	\item DELETE : Supprimer une ressources
\end{itemize}

\textbf{Remarque :} C'est une convention, on peut tout à fait faire le travail de l'une avec l'autre.\newline


On va utiliser \textbf{POST}, pour faire une méthode pour ajouter un étudiant dans la liste. On commence à écrire une méthode qui prend en argument un étudiant. Les attributs de l'étudiant sont directement passé dans l'URL. L'entrypoint est \textbf{addEtudiant}. Via SoapUI, on peut ajouter un étudiant.

\begin{minipage}{0.5\textwidth}
\includegraphics[width=8cm]{images/P7_A.png}
\captionof{figure}{Ajout de l'étudiant 5}
\label{fig:figure}
\end{minipage}
\hspace{1cm}
\begin{minipage}{0.5\textwidth}
\includegraphics[width=8cm]{images/P7_B.png}
\captionof{figure}{Liste de tous les étudiants}
\label{fig:figure}
\vspace{1cm}
\end{minipage}

On peut dans le même esprit écrire des méthodes pour modifier et supprimer un éléve avec les requetes PUT et DELETE.

\begin{minipage}{0.5\textwidth}
\includegraphics[width=8cm]{images/P8_A.png}
\captionof{figure}{Modification de l'étudiant 3}
\label{fig:figure}
\end{minipage}
\hspace{1cm}
\begin{minipage}{0.5\textwidth}
\includegraphics[width=8cm]{images/P8_B.png}
\captionof{figure}{Liste de tous les étudiants}
\label{fig:figure}
\vspace{1cm}
\end{minipage}

\begin{minipage}{0.5\textwidth}
\includegraphics[width=8cm]{images/P9_A.png}
\captionof{figure}{Suppression de l'étudiant 1}
\label{fig:figure}
\end{minipage}
\hspace{1cm}
\begin{minipage}{0.5\textwidth}
\includegraphics[width=8cm]{images/P9_B.png}
\captionof{figure}{Liste de tous les étudiants}
\label{fig:figure}
\vspace{1cm}
\end{minipage}


\textcolor{red}{\textbf{Remarque} : Les changements sont persistant selon la durée de vie du processus du SpringBootApplication. Son arrêt entraine la suppression de tout ce qu'il y a dans la RAM}.

\section{Persistance avec base de donnée H2}

La partie suivante consiste à utiliser le controleur JPA pour faire l'interfaçage entre le Java et le SGBD \textbf{H2}. \textbf{H2} est un SGBD écrit entièrmement en Java. Il est destiné au devs et non à la production. De ce point de vue, H2 à des similitudes avec \textbf{SQLite}. Cette partie à pu marcher avec l'IDE \textbf{InteliJ} (Je ne sais pas pourquoi).

\subsection{Installation}

On va générer un nouveau projet. On reprend les mêmes êtapes précédentes à la diférence des dépendances. On va rajouter les dépéndances \textbf{H2 Database} et \textbf{Spring Data JPA}.

\textcolor{red}{\textbf{Remarque} : Pour mon cas, j'ai du utiliser JDK 21 avec Java 17 pour m'assurer que le projet soit compatible avec Maven}.\newline

On va générer deux nouveau package dans \textbf{com.example.h2demo\_tp3}. Le package \textbf{entities} qui représente les entités de notre application. Enfin, le package \textbf{repository}. Ce package permet d'organiser les accès à la base de donnée. 

\subsection{Micro-services avec Hibernate}

Pour écrire notre nouvelle API REST. On va d'abord importer les bibliothèques \textbf{Hibernate} via les modules  \textbf{jakarta.persistance} pour manipuler la base de donnée \textbf{H2}. Il est à noter qu'il existe d'autres contrôleur que \textbf{Hibernate}.\newline

On commence par écrire une classe Java clasique avec ces attributs et méthodes. Dans notre cas, on s'intéresse à un Adhérent. On doit préciser lequel des attributs est une clé primaires avec l'annotation \textbf{@Id}. De  même, la classe \textbf{Adherent} doit être annoté par \textbf{@Entity} pour qu'une table soit généré.\newline

On va ensuite définir une interface AdherentRepository. Cette interface va hériter de \textbf{JpaRepository<Adherent,Long>}. Le 1er paramètre doit être notre entité(Classe) tandis que le deuxième doit être le type de la clé primaire. La déclaration de cette interface va nous fournir un ensemble  d'opérations \textbf{CRUD} (Create,Read,Update,Delete) de base sur nos données. L'avantage est que n'avons plus besoin de générer des méthodes pour manipuler nos entités (Comme vu auparavant). Ces méthodes sont directement généré.\newline

\begin{lstlisting}
package com.example.h2demo_tp3.repository;


import com.example.h2demo_tp3.entities.Adherent;
import org.springframework.data.jpa.repository.JpaRepository;

public interface AdherentRepository extends JpaRepository<Adherent, Long>{}
\end{lstlisting}

Un exemple de ces bénéfices est la classe [Nom\_projet]Application. Içi, on va pouvoir directement enregistrer des instances d'Adhérent dans la base de donnée en important l'interface. Spring va exécuter la méthode \textbf{runner de type CommandLineRunner} pendant l'exécution. On va ainsi dès le démarage enregistrer 4 adhérents.

\textbf{Rmq : La synataxe \textbf{return -> args {} est une fonction lambda (Anonyme) dans Java}}.\newline

\begin{figure}[H]
	\centering
    \includegraphics[width=15cm]{images/P10_A.png}
    \caption{Application.java}
\end{figure}

Pour voir l'ensemble des ces résultats, on doit éditer le fichier \textbf{properties} dans ressources. De cette manière :
\begin{lstlisting}
1 spring.application.name=H2demo_TP3
2 server.port=9191
3 spring.datasource.url=jdbc:h2:mem:adherent
4 spring.h2.console.enabled=true
\end{lstlisting}

La ligne 3, précise l'URL de la JDBC. La ligne 4 active le mode console. C'est-à-dire le GUI du SGBD H2.\newline

Après avoir lancé le serveur, on peut consulter l'interface de H2. Via l'url \href{http://localhost:9191/h2-console}. 


\begin{minipage}{0.5\textwidth}
\includegraphics[width=8cm]{images/P11_A.png}
\captionof{figure}{On peut appuyer sur \textbf{Test connection} pour tester la connection}
\label{fig:figure}
\end{minipage}
\hspace{1cm}
\begin{minipage}{0.5\textwidth}
\includegraphics[width=8cm]{images/P9_B.png}
\captionof{figure}{Interface GUI de H2}
\label{fig:figure}
\vspace{1cm}
\end{minipage}

\vspace{1cm}
On peut visualiser tous les adhérent avec une requête SQL. 


\begin{figure}[H]
	\centering
    \includegraphics[width=5cm]{images/P11_C.png}
    \caption{Application.java}
\end{figure}


\end{document}






























